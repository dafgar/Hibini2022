\section{Заключение}
\subsection{Выводы и рекомендации по маршруту}
Нами пройден полностью автономный линейный маршрут по интересовавшему нас району.
Посещены все запланированные достопримечательности.

Для групп, планирующих более спортивный маршрут, рекомендуем выход к пер. Юмъекорр через ущ. Аку-Аку,
и выход из долины руч. Меридиональный к р. Малой Белой либо через связку
перевалов Импульс (1А) - Почтальон (н/к) - Медвежий Лог (н/к), либо подъём по р. Гольцовка
и переход через один из перевалов Арсенина (н/к - 1А).

\subsubsection*{Рекомендации по снаряжению}
В отчете многократно упоминались переходы ручьев и рек.
В действительности в это время года практически любую водную преграду можно перейти по обливным камням в русле,
но без хорошей мембранной обуви придется много времени тратить на переобувание в бродовую и обратно,
либо ходить с мокрыми ногами.

\subsection{Благодарности}
Группа выражает признательность всем участвовавшим в организации и поддержке нашего похода.

Отдельная благодарность:

Олегу Власенко (\url{https://vk.com/vlasenko_maps}), за карты, существенно облегчившие планирование маршрута;

МКК ФСТ-ОТМ, за то что выпустили нас в поход в последнюю остающуюся перед походом выпускную среду;

Наташе Ключник, за согласие побыть координатором в Москве;

группе туристов т/к ВШЭ, за демонстрацию того, как можно красиво сверстать отчёт о походе
(МК \textnumero 1/1-222 ФСТ-ОТМ от 2021 года) и пример тематической разбивки туристского отчёта.
