\section{Организация и проведение похода}
\subsection{Идея похода}
Идея похода по Хибинам возникла в рамках более широкой идеи автомобильного путешествия из Москвы на Кольский полуостров.
Небольшой пешеходный поход длительностью около недели отлично вписывался расписание, и позволил участникам
всё-таки сходить в "настоящий" поход в этом сезоне. Кроме того не возникало проблемы с тем, как добираться до района похода.

Хибинский горный массив знаком обоим участникам. Один из них, родом из Мурманска, давно не был в Хибинах и был не прочь
пройти по ним спортивный поход. У второго участника, история романтическая - в 2015, после того как он c женой
в ходе однодневного ПВД решил не рисковать и не дошёл до оз. Академического, началось его увлечение организованным туризмом.

\subsection{Планирование маршрута}
Поскольку ни у одного участника не было опыта 1ПУ, то в маршрут вошли только некатегорийные перевалы
(траверс пер. Исток (1А) по вездеходной дороге сравним по сложности с прохождением пер. Умбозёрского (н/к)).
С одной стороны, это несколько уменьшило количество вариантов, с другой - мы больше перемещались и больше посмотрели.
В итоге маршрут получился обзорным по всему массиву (почти по всему, всё-таки только 8 дней).
Из достопримечательностей обязательно хотелось посмотреть ущ. Аку-Аку, пройтись вдоль р. Малой Белой,
побывать на оз. Гольцовом и Академическом. Старт и финиш, в связи с отсутствием необходимости попасть в
конкретную часть Хибин, логично было организовать в местах с ж/д и автотранспортом.
В нашем случае это были оз. Малый Вудьявр (доступно на городском такси из Кировска) и пл. Имандра электрички,
с западной стороны массива. По разбивке по времени прохождения препятствий выходило, что предпочтительнее стартовать
с пл. Имандра.

В итоге получился несложный линейный маршрут с несколькими перевалами и передвижением по долинам ручьёв и рек.

\subsection{Аварийные варианты}
В силу компактности района Хибин, на всём протяжении маршрута есть возможность выйти к объектам
человеческой инфраструктуры менее чем за сутки через н/к перевалы. В западной части маршрута
(до пер. Петрелиуса Западный) аварийные сходы были запланированы вниз по долинам основных рек и ручьёв к линии
железной дороги (р. Гольцовка, руч. Нефелиновый, руч. Медвежий Лог, р. Малая Белая). В остальной части маршрута
аварийный сход предполагался по пути подъёма или спуска: либо в сторону базы КСС, либо в сторону оз. Малый Вудьявр.

Поскольку строгого ограничения на продолжительность похода не было, то запасных вариантов прохождения маршрута не
составлялось. В случае нехватки времени, был вариант дойти до пер. Умбозёрский, далее вернуться до КСС
и по автомобильной дороге выйти к оз. Малый Вудьявр.

\subsection{Обеспечение безопасности на маршруте}
Участники группы имеют специальное и базовое туристское образование и опыт спортивных походов, регулярно в течение года ходят в ПВД и участвуют в соревнованиях по туристскому ориентированию.

На время нахождения на маршруте группа была зарегистрирована в территориальном органе МЧС России.

В Москве у группы был координатор, имеющий информацию о маршруте группы, контакты для связи с МЧС, МКК.
Ежедневно (при наличии связи) группа связывалась по мобильным телефонам с дежурным МЧС и координатором.

\subsection{Оформление документов}
\subsubsection{МЧС}
Согласно приказу МЧС России \textnumero 42 от 30.01.2019 (утвержденному в соответствии с
постановлением правительства РФ \textnumero 252 от 03.03.2017), начиная с 09.03.2019, туристские группы должны
проинформировать территориальный орган МЧС России о предполагаемом маршруте не позже чем за 10 рабочих дней
до выхода на маршрут (о чём также напоминается в маршрутной книжке нового образца).
Подача уведомления осуществляется на сайте МЧС России \url{https://forms.mchs.gov.ru},
после чего с вами связывается дежурный и сообщает регистрационный номер группы.
Также дежурный сообщает, что связь на маршруте осуществляется ежедневно (чего по нашему опыту ранее не было).
В случае Мурманского МЧС это действительно оказалось так. Когда мы очутились ближе к цивилизации,
через пару дней без связи, на телефоне было оставлено несколько голосовых сообщений.
А спасатель на базе КСС, связавшись по радиостанции с дежурным, сказал, что нас уже потеряли.
Впрочем он сам объяснил дежурному, что связи в горах нет. Почему дежурному об этом неизвестно - непонятно.

\subsubsection{Национальный парк Хибины}
Как известно из истории, сначала туристы (и все прочие неравнодушные) подписывали петицию,
чтобы горнодобытчикам не дали срыть Хибинский горный массив, а затем туристы подписывали петицию,
чтобы в организованном национальном парке оставили возможность для посещения самостоятельных туристских групп.
В 2018 году на территории Хибин был организован национальный парк "Хибины".
В ответ на запрос к администрации национального парка (e-mail: tur@laplandzap.ru), нам сообщили,
что ограничений на посещение территории парка нет, но надо зарегистрироваться в МЧС
и сообщить администрации регистрационный номер группы и маршрут следования.
Также предупредили \textbf{\color{red}о запрете розжига костров}. Ни одного инспектора заповедника на маршруте нам не встретилось.
Через пару дней после окончания похода нам позвонили и поинтересовались как дела.

\subsection{Картографическое обеспечение}
При планировании маршрута использовались картографические базы данных OpenStreetMap и база,
созданная Олегом Власенко (\url{https://vk.com/vlasenko_maps}, далее \textit{VM}).
Для работы с первой использовалась программа \textit{Viking},
со второй - \textit{QMapShack}. База \textit{VM} удобна тем, что в ней для всех перевалов указаны их категории трудности
и удобно настроены уровни отображения объектов, так что отображение не загромождено деталями.

Поскольку на тот момент в продаже бумажная версия карты \textit{VM} отсутствовала,
мы импортировали из базы \textit{VM} слои с перевалами и дорогами в \textit{QGIS},
там наложили эти слои на слой отображения генштабовки и создали атлас из 6 страниц,
покрывающих район Хибин. Далее распечатали в цвете на 3-х листах, заламинировали и получили комплект карт.

В качестве GPS-навигатора в походе использовался Garmin GPSMAP 64.
Быстрый и эргономичный, но в то же время очень прожорливый - комплекта из 2-х АА батареек хватало чуть более чем на день.
На навигатор были установлены карты \textit{VM} и сборка \textit{OpenStreetMap}
от \textit{maptourist} (\url{https://maptourist.org}, на момент написания отчёта сайт не открывается),
использовались преимущественно первые. Также в качестве второго GPS-навигатора (и дублирующего записывающего)
использовался водонепроницаемый китайский смартфон с картами \textit{OSM} для \textit{LocusMap}.

Обработка и анализ записанных GPS-треков были произведены в программе \textit{Viking}.

\subsection{Материальное и продуктовое обеспечение группы}
\subsubsection{Групповое туристское снаряжение}
\begin{itemize}
\item палатка 2-ка
\item газовая горелка-джет и дополнительный кан на 1.2 литра
\item аптечка
\item ремнабор
\item GPS-навигатор и набор бумажных карт
\item фотоаппарат
\item термометр
\item часы
\end{itemize}

\subsubsection{Личное специальное снаряжение}
\begin{itemize}
\item каска
\item пара треккинговых палок
\item кордалет, 7 м
\item фальшфейер
\item карабин
\end{itemize}

Каски брались прежде всего для преодоления горных ручьёв, в частности из-за ужасающего вида руч. Ферсмана на фотографиях.
По факту руч. Ферсмана был пересечён по мосту, а каски пригодились на пер. Медвежий Лог для защиты от пронизывающего ветра.

\subsubsection{Продуктовое обеспечение группы}
Масса продуктов в день на человека составляла 600 г. Голодать не пришлось.

Стандартный пищевой распорядок дня выглядел так:
\begin{itemize}
\item утром: сладкая молочная каша + чай + сладкое + карпит
\item днём: суп + сыр + колбаса + чай
\item вечером: мясная каша + чай + сладкое
\end{itemize}

\subsubsection{Итоговая масса}
Итоговая носимая масса составила 20.8 кг на человека.

\subsection{Наша заброска}
Старт похода был запланирован с пл. Имандра ж/д. Поскольку ж/д в г. Кировске нет, а электричка идёт рано утром,
то заночевать мы решили в г. Апатиты, где есть ж/д вокзал. 04.08 мы доехали на личном автомобиле до г. Апатиты,
где заселились в гостиницу "Аметист". Гостиницу бронировали не заранее, а днём ранее
(при помощи \url{https://travel.yandex.ru}), поэтому стоимость номера (2790 р/ночь) неприятно удивила.
Ночевать решили в гостинице, чтобы постираться, искупаться и выспаться.

Машину оставили на единственной платной стоянке в городе (ул. Строителей, координаты:
N$67.5595^{\circ}$, E$33.3929^{\circ}$),
которую также нашли через интернет, за $130$ рублей в сутки

Утром 05.08 сначала попытались вызвать такси через приложение яндекс.такси, но оно показывало только одну машину,
да и та не вызывалась. Поэтому взяли номер телефона такси на стойке регистрации в гостинице и вызвали местное,
на котором и доехали успешно до вокзала. Отправление электрички в 7:44 (ходит только в пт/сб/вс/пн),
ехать до пл. Имандра 40 минут, билеты продают непосредственно в электричке и только за наличные.
Стоимость проезда по маршруту вокзал Апатит - пл. Имандра 99 рублей. В 8:24 высадились на пл. Имандра.

По завершении маршрута у оз. Малый Вудьявр, при помощи приложения яндекс.такси и мерцающего интернета,
вызвали такси на котором доехали в г. Апатиты до стоянки личного автомобиля.
Во время поездки на том же сайте забронировали номер в санатории Изовела (ул. Победы, 29А, 2091 р/ночь),
где переночевали и на следующий день тронулись дальше в путь по Кольскому полуострову.
